

\chapter{Case study: word play  |  文字游戏}
\label{wordplay}

This chapter presents the second case study, which involves
solving word puzzles by searching for words that have certain
properties.  For example, we'll find the longest palindromes
in English and search for words whose letters appear in
alphabetical order.  And I will present another program development
plan: reduction to a previously solved problem.

这一章将介绍第二个案例研究,即通过查找具有特定属性的单词来解答字谜游戏。
例如,我们将找出英文中最长的回文单词,以及字符按照字符表顺序出现的单词。
另外,我还将介绍另一种程序开发方法:简化为之前已解决的问题。

\section{Reading word lists  |  读取单词列表}
\label{wordlist}

For the exercises in this chapter we need a list of English words.
There are lots of word lists available on the Web, but the one most
suitable for our purpose is one of the word lists collected and
contributed to the public domain by Grady Ward as part of the Moby
lexicon project (see \url{http://wikipedia.org/wiki/Moby_Project}).  It
is a list of 113,809 official crosswords; that is, words that are
considered valid in crossword puzzles and other word games.  In the
Moby collection, the filename is {\tt 113809of.fic}; you can download
a copy, with the simpler name {\tt words.txt}, from
\url{http://thinkpython2.com/code/words.txt}.
\index{Moby Project}  \index{crosswords}

为了完成本章的习题,我们需要一个英语单词的列表。
网络上有许多单词列表,但是最符合我们目的列表之一是由 Grady
Ward收集并贡献给公众的列表,这也是Moby词典项目的一部分
(见:\url{http://wikipedia.org/wiki/Moby_Project})。
它由 113,809 个填字游戏单词组成,即在填字游戏以及其它文字游戏中被认为有效的单词。
在 Moby 集合中,该列表的文件名是 113809of.fic ;你可以从 \url{http://thinkpython.com/code/words.txt} 下载一个拷贝,文件名已被简化为 words.txt。
\index{Moby Project}  \index{crosswords}

This file is in plain text, so you can open it with a text
editor, but you can also read it from Python.  The built-in
function {\tt open} takes the name of the file as a parameter
and returns a {\bf file object} you can use to read the file.

该文件是纯文本, 因此你可以用一个文本编辑器打开它, 但是你也可以从Python中读取它。 内建函数 \li{open} 接受文件名作为形参, 并返回一个 {\em 文件对象} (file object), 你可以使用它读取该文件。

\index{open function}  \index{function!open}
\index{plain text}  \index{text!plain}
\index{object!file}  \index{file object}

\begin{lstlisting}
>>> fin = open('words.txt')
\end{lstlisting}

%
{\tt fin} is a common name for a file object used for input.  The file
object provides several methods for reading, including {\tt readline},
which reads characters from the file until it gets to a newline and
returns the result as a string: \index{readline method}
\index{method!readline}

\li{`fin} 是输入文件对象的一个常用名。 该文件对象提供了几个读取方法,
包括 \li{readline}, 其从文件中读取字符直到碰到新行, 并将结果作为字符串返回:
\index{method!readline}

\begin{lstlisting}
>>> fin.readline()
'aa\r\n'
\end{lstlisting}

%
The first word in this particular list is ``aa'', which is a kind of
lava.  The sequence \verb"\r\n" represents two whitespace characters,
a carriage return and a newline, that separate this word from the
next.

在此列表中,第一个单词是 ``aa'', 它是一类熔岩的名称。 序列 \li{\r\n} 代表两个空白字符, 回车和换行, 它们将这个单词和下一个分开。

The file object keeps track of where it is in the file, so
if you call {\tt readline} again, you get the next word:

此文件对象跟踪它在文件中的位置,
所以如果你再次调用 \li{readline},你获得下一个单词:

\begin{lstlisting}
>>> fin.readline()
'aah\r\n'
\end{lstlisting}

%
The next word is ``aah'', which is a perfectly legitimate
word, so stop looking at me like that.
Or, if it's the whitespace that's bothering you,
we can get rid of it with the string method {\tt strip}:

下一个单词是 ``aah'',不要惊讶,它是一个完全合法的单词。
或者, 如果空格困扰了你, 我们可以用字符串方法 \li{strip} 删掉它:
\index{strip method}  \index{method!strip}
\index{strip 方法}  \index{方法!strip}


\begin{lstlisting}
>>> line = fin.readline()
>>> word = line.strip()
>>> word
'aahed'
\end{lstlisting}

%
You can also use a file object as part of a {\tt for} loop.
This program reads {\tt words.txt} and prints each word, one
per line:

你也可以将文件对象用做 \li{for} 循环的一部分。
此程序读取 \li{words.txt} 并打印每个单词,每行一个:
\index{open function}  \index{function!open}

\begin{lstlisting}
fin = open('words.txt')
for line in fin:
    word = line.strip()
    print(word)
\end{lstlisting}

%

\section{Exercises  |  练习}

There are solutions to these exercises in the next section.
You should at least attempt each one before you read the solutions.

\hyperref[search]{下一节}将给出了这些习题的答案。
在你看答案之前,应该试着解答一下。

\begin{exercise}
Write a program that reads {\tt words.txt} and prints only the
words with more than 20 characters (not counting whitespace).

编程写一个程序,使得它可以读取 \li{words.txt} , 然后只打印出那些长度超过 20 个字符的单词 (不包括空格)。
\index{whitespace}

\end{exercise}

\begin{exercise}

In 1939 Ernest Vincent Wright published a 50,000 word novel called
{\em Gadsby} that does not contain the letter ``e''.  Since ``e'' is
the most common letter in English, that's not easy to do.

1939年,Ernest Vincent Wright 出版了一本名为 《Gadsby》 的小说,
该小说里完全没有使用字符 ``e''。 由于 ``e'' 是英文中最常用的字符, 因此写出这本书并不容易。

In fact, it is difficult to construct a solitary thought without using
that most common symbol.  It is slow going at first, but with caution
and hours of training you can gradually gain facility.

事实上,不使用这个最常用的字符来构建一个简单的想法都是很难的。
开始进展缓慢,但是经过有意识的、长时间的训练,你可以逐渐地熟练。

All right, I'll stop now.

好啦,不说题外话了,我们开始编程练习。

Write a function called \verb"has_no_e" that returns {\tt True} if
the given word doesn't have the letter ``e'' in it.

写一个叫做 \li{has_no_e} 的函数, 如果给定的单词中不包含字符 ``e'',返回 \li{True} 。

Modify your program from the previous section to print only the words
that have no ``e'' and compute the percentage of the words in the list
that have no ``e''.

修改上一节中的程序, 只打印不包含 ``e'' 的单词,并且计算列表中不含 ``e'' 单词的比例。
\index{lipogram}

\end{exercise}


\begin{exercise}

Write a function named {\tt avoids}
that takes a word and a string of forbidden letters, and
that returns {\tt True} if the word doesn't use any of the forbidden
letters.

编写一个名为 \li{avoids} 的函数,接受一个单词和一个指定禁止使用字符的字符串,
如果单词中不包含任意被禁止的字符,则返回 \li{True} 。

Modify your program to prompt the user to enter a string
of forbidden letters and then print the number of words that
don't contain any of them.
Can you find a combination of 5 forbidden letters that
excludes the smallest number of words?

修改你的程序,提示用户输入一个禁止使用的字符,然后打印不包含这些字符的单词的数量。
你能找到一个5个禁止使用字符的组合,使得其排除的单词数目最少么?

\end{exercise}



\begin{exercise}

Write a function named \verb"uses_only" that takes a word and a
string of letters, and that returns {\tt True} if the word contains
only letters in the list.  Can you make a sentence using only the
letters {\tt acefhlo}?  Other than ``Hoe alfalfa?''

编写一个名为 \li{uses_only} 的函数,接受一个单词和一个字符串。
如果该单词只包括此字符串中的字符,则返回 \li{True}。
你能只用 \li{"acefhlo"} 这几个字符造一个句子么? 除了 ``Hoe alfalfa'' 外。

\end{exercise}


\begin{exercise}

Write a function named \verb"uses_all" that takes a word and a
string of required letters, and that returns {\tt True} if the word
uses all the required letters at least once.  How many words are there
that use all the vowels {\tt aeiou}?  How about {\tt aeiouy}?

编写一个名为 \li{uses_all} 的函数,接受一个单词和一个必须使用的字符组成的字符串。
如果该单词包括此字符串中的全部字符至少一次,则返回 \li{True}。
你能统计出多少单词包含了所有的元音字符 \li{aeiou}吗?如果换成 \li{aeiouy} 呢?

\end{exercise}


\begin{exercise}

Write a function called \verb"is_abecedarian" that returns
{\tt True} if the letters in a word appear in alphabetical order
(double letters are ok).
How many abecedarian words are there?

编写一个名为 \li{is_abecedarian} 的函数, 如果单词中的字符以字符表的顺序出现(允许重复字符),则返回 \li{True} 。
有多少个具备这种特征的单词?

\index{abecedarian}

\end{exercise}



\section{Search  ||  搜索}
\label{search}
\index{search pattern}  \index{pattern!search}

All of the exercises in the previous section have something
in common; they can be solved with the search pattern we saw
in Section~\ref{find}.  The simplest example is:

前一节的所有习题有一个共同点;都可以用在~\ref{find}一节中看到的搜索模式解决。

\begin{lstlisting}
def has_no_e(word):
    for letter in word:
        if letter == 'e':
            return False
    return True
\end{lstlisting}
%
The {\tt for} loop traverses the characters in {\tt word}.  If we find
the letter ``e'', we can immediately return {\tt False}; otherwise we
have to go to the next letter.  If we exit the loop normally, that
means we didn't find an ``e'', so we return {\tt True}.

\li{for} 循环遍历 \li{word} 中的字符。
如果我们找到字符 \li{"e"},那么我们可以马上返回 \li{False} ;
否则我们必须检查下一个字符。
如果我们正常退出循环,就意味着我们没有找到一个“e”, 所以我们返回 \li{True} 。

\index{traversal}

\index{in operator}  \index{operator!in}

You could write this function more concisely using the {\tt in}
operator, but I started with this version because it
demonstrates the logic of the search pattern.

你也可以用 \li{in} 操作符简化上述函数,但之所以一开始写成这样,是因为它展示了搜索模式的逻辑。

\index{generalization}

{\tt avoids} is a more general version of \verb"has_no_e" but it
has the same structure:

\li{avoid} 是一个更通用的 \li{has_no_e} 函数,但是结构是相同的:

\begin{lstlisting}
def avoids(word, forbidden):
    for letter in word:
        if letter in forbidden:
            return False
    return True
\end{lstlisting}

%
We can return {\tt False} as soon as we find a forbidden letter;
if we get to the end of the loop, we return {\tt True}.

一旦我们找到一个禁止使用的字符,我们返回 \li{False} ;
如果我们到达循环结尾,我们返回 \li{True} 。

\verb"uses_only" is similar except that the sense of the condition
is reversed:

除了检测条件相反以外,下面 \li{uses_only} 函数与上面的函数很像:

\begin{lstlisting}
def uses_only(word, available):
    for letter in word:
        if letter not in available:
            return False
    return True
\end{lstlisting}
%
Instead of a list of forbidden letters, we have a list of available
letters.  If we find a letter in {\tt word} that is not in
{\tt available}, we can return {\tt False}.

这里我们传入一个允许使用字符的列表,而不是禁止使用字符的列表。
如果我们在 \li{word} 中找到一个不在 \li{available} 中的字符,我们就可以返回 \li{False} 。

\verb"uses_all" is similar except that we reverse the role
of the word and the string of letters:

除了将 \li{word} 与所要求的字符的角色进行了调换之外,
下面的 \li{uses_all} 函数也是类似的。

\begin{lstlisting}
def uses_all(word, required):
    for letter in required:
        if letter not in word:
            return False
    return True
\end{lstlisting}

%
Instead of traversing the letters in {\tt word}, the loop
traverses the required letters.  If any of the required letters
do not appear in the word, we can return {\tt False}.

该循环遍历需要的字符,而不是遍历 \li{word} 中的字符。如果任何要求的字符没出现在单词中, 则我们返回 \li{False} 。

\index{traversal}

If you were really thinking like a computer scientist, you would
have recognized that \verb"uses_all" was an instance of a
previously solved problem, and you would have written:

如果你真的像计算机科学家一样思考,你可能已经意识到 \li{uses_all} 是前面已经解决的问题的一个实例, 你可能会写成:

\begin{lstlisting}
def uses_all(word, required):
    return uses_only(required, word)
\end{lstlisting}
%
This is an example of a program development plan called {\bf
  reduction to a previously solved problem}, which means that you
recognize the problem you are working on as an instance of a solved
problem and apply an existing solution.  \index{reduction to a
  previously solved problem} \index{development plan!reduction}

这是一种叫做 {\em 简化为之前已解决的问题} (reduction to a
previously solved problem) 的程序开发方法的一个示例,
也就是说,你认识到当前面临的问题是之前已经解决的问题的一个实例,
然后应用了已有的解决方案。


\section{Looping with indices  |  使用索引进行循环}
\index{looping!with indices}  \index{index!looping with}

I wrote the functions in the previous section with {\tt for}
loops because I only needed the characters in the strings; I didn't
have to do anything with the indices.

前一节我用 \li{for} 循环来编写函数,因为我只需要处理字符串中的字符;
我不必用索引做任

For \verb"is_abecedarian" we have to compare adjacent letters,
which is a little tricky with a {\tt for} loop:

对于下面的 \li{is_abecedarian} , 我们必须比较邻接的字符, 用 \li{for} 循环来写的话有点棘手。

\begin{lstlisting}
def is_abecedarian(word):
    previous = word[0]
    for c in word:
        if c < previous:
            return False
        previous = c
    return True
\end{lstlisting}

An alternative is to use recursion:

一种替代方法是使用递归:

\begin{lstlisting}
def is_abecedarian(word):
    if len(word) <= 1:
        return True
    if word[0] > word[1]:
        return False
    return is_abecedarian(word[1:])
\end{lstlisting}

Another option is to use a {\tt while} loop:

另一中方法是使用 \li{while} 循环:

\begin{lstlisting}
def is_abecedarian(word):
    i = 0
    while i < len(word)-1:
        if word[i+1] < word[i]:
            return False
        i = i+1
    return True
\end{lstlisting}

%
The loop starts at {\tt i=0} and ends when {\tt i=len(word)-1}.  Each
time through the loop, it compares the $i$th character (which you can
think of as the current character) to the $i+1$th character (which you
can think of as the next).

循环起始于 \li{i=0} , \li{i=len(word)-1} 时结束。
每次循环,函数会比较第 $i$ 个字符(可以将其认为是当前字符)
和第 $i+1$ 个字符(可以将其认为是下一个字符)。

If the next character is less than (alphabetically before) the current
one, then we have discovered a break in the abecedarian trend, and
we return {\tt False}.

如果下一个字符比当前的小(字符序靠前),
那么我们在递增趋势中找到了断点,即可返回 \li{False} 。

If we get to the end of the loop without finding a fault, then the
word passes the test.  To convince yourself that the loop ends
correctly, consider an example like \verb"'flossy'".  The
length of the word is 6, so
the last time the loop runs is when {\tt i} is 4, which is the
index of the second-to-last character.  On the last iteration,
it compares the second-to-last character to the last, which is
what we want.

如果到循环结束时我们也没有找到一点错误,那么该单词通过测试。
为了让你相信循环正确地结束了,我们用 \li{'flossy'} 这个单词来举例。
它的长度为 6,因此最后一次循环运行时,\li{i} 是 4,这是倒数第 2 个字符的索引。
最后一次迭代时,函数比较倒数第二个和最后一个字符,这正是我们希望的。
\index{palindrome}

Here is a version of \verb"is_palindrome" (see
Exercise~\ref{palindrome}) that uses two indices; one starts at the
beginning and goes up; the other starts at the end and goes down.

下面是 \li{is_palindrome} 函数的一种版本(详见 练习~\ref{palindrome} )
,其中使用了两个索引;一个从最前面开始并往前上, 另一个从最后面开始并往下走。

\begin{lstlisting}
def is_palindrome(word):
    i = 0
    j = len(word)-1

    while i<j:
        if word[i] != word[j]:
            return False
        i = i+1
        j = j-1

    return True
\end{lstlisting}

Or we could reduce to a previously solved
problem and write:

或者,我们可以把问题简化为之前已经解决的问题,这样来写:

\index{reduction to a previously solved problem}
\index{development plan!reduction}

\begin{lstlisting}
def is_palindrome(word):
    return is_reverse(word, word)
\end{lstlisting}
%
Using \verb"is_reverse" from Section~\ref{isreverse}.

使用 \ref{isreverse}~节 中描述的 \li{is_reverse}


\section{Debugging  |  调试}
\index{debugging}  \index{testing!is hard}  \index{program testing}

Testing programs is hard.  The functions in this chapter are
relatively easy to test because you can check the results by hand.
Even so, it is somewhere between difficult and impossible to choose a
set of words that test for all possible errors.

程序测试很困难。本章中介绍的函数相对容易测试,因为你可以手工检查结果。
即使这样,选择一可以测试所有可能错误的单词集合,是很困难的,介于困难和不可能之间。

Taking \verb"has_no_e" as an example, there are two obvious
cases to check: words that have an `e' should return {\tt False}, and
words that don't should return {\tt True}.  You should have no
trouble coming up with one of each.

以 \li{has_no_e} 为例,有两个明显的用例需要检查:
含有 `e' 的单词应该返回 \li{False},不含的单词应该返回 \li{True} 。
你应该可以很容易就能想到这两种情况。

Within each case, there are some less obvious subcases.  Among the
words that have an ``e'', you should test words with an ``e'' at the
beginning, the end, and somewhere in the middle.  You should test long
words, short words, and very short words, like the empty string.  The
empty string is an example of a {\bf special case}, which is one of
the non-obvious cases where errors often lurk.

在每个用例中,还有一些不那么明显的子用例。
在含有 ``e'' 的单词中,你应该测试 ``e'' 在开始、结尾以及在中间的单词。
你还应该测试长单词、短单词以及非常短的单词,如空字符串。
空字符串是一个 {\em 特殊用例} (special case) ,及一个经常出现错误的不易想到的用例。
\index{special case}

In addition to the test cases you generate, you can also test
your program with a word list like {\tt words.txt}.  By scanning
the output, you might be able to catch errors, but be careful:
you might catch one kind of error (words that should not be
included, but are) and not another (words that should be included,
but aren't).

除了你生成的测试用例,你也可以用一个类似 \li{words.txt} 中的单词列表测试你的程序。 通过扫描输出,你可能会捕获错误,但是请注意:
你可能捕获一类错误(包括了不应该包括的单词)却没能捕获另一类错误(没有包括应该包括的单词)。

In general, testing can help you find bugs, but it is not easy to
generate a good set of test cases, and even if you do, you can't
be sure your program is correct.
According to a legendary computer scientist:
\index{testing!and absence of bugs}

\begin{quote}
Program testing can be used to show the presence of bugs, but never to
show their absence!

--- Edsger W. Dijkstra
\end{quote}
\index{Dijkstra, Edsger}

一般来讲,测试能帮助你找到错误, 但是生成好的测试用例并不容易,
并且即便你做到了,你仍然不能保证你的程序是正确的。正如一位传奇计算机科学家所说:
\index{testing!and absence of bugs}
\begin{quote}
{\bf 程序测试能用于展示错误的存在,但是无法证明不存在错误!}

--- Edsger W. Dijkstra
\end{quote}
\index{Dijkstra, Edsger}


\section{Glossary  |  术语表}

\begin{description}

\item[file object:] A value that represents an open file.
\index{file object}
\index{object!file}

\item[文件对象(file object):] 代表打开文件的变量。
\index{file object}
\index{object!file}

\item[reduction to a previously solved problem:] A way of solving a
  problem by expressing it as an instance of a previously solved
  problem.
  \index{reduction to a previously solved problem}
  \index{development plan!reduction}

\item[简化为之前已经解决的问题:] 通过把未知问题简化为已经解决的问题来解决问题的方法。
\index{reduction to a previously solved problem}
\index{development plan!reduction}

\item[special case:] A test case that is atypical or non-obvious
(and less likely to be handled correctly).
\index{special case}

\item[特殊用例(special case):] 一种不典型或者不明显的测试用例(而且很可能无法正确解决的用例)。
\index{special case}

\end{description}


\section{Exercises}

\begin{exercise}
\index{Car Talk}  \index{Puzzler}  \index{double letters}

This question is based on a Puzzler that was broadcast on the radio
program {\em Car Talk}
(\url{http://www.cartalk.com/content/puzzlers}):

这个问题基于广播节目 *《Car Talk》* (http://www.cartalk.com/content/puzzlers)上介绍的一个字谜:

\begin{quote}
Give me a word with three consecutive double letters. I'll give you a
couple of words that almost qualify, but don't. For example, the word
committee, c-o-m-m-i-t-t-e-e. It would be great except for the `i' that
sneaks in there. Or Mississippi: M-i-s-s-i-s-s-i-p-p-i. If you could
take out those i's it would work. But there is a word that has three
consecutive pairs of letters and to the best of my knowledge this may
be the only word. Of course there are probably 500 more but I can only
think of one. What is the word?
\end{quote}

\begin{quote}
    找出一个包含三个连续双字符的单词。我将给你一系列几乎能够符合条件但实际不符合的单词。
    比如,committee这个单词,c-o-m-m-i-t-t-e-e。
    如果中间没有i的话,就太棒了。
    或者Mississippi这个单词: M-i-s-s-i-s-s-i-p-p-i。假如将这些i剔除出去,就会符合条件。但是确实存在一个包含三个连续的单词对,
    而且据我了解,它可能是唯一符合条件的单词。
    当然也可能有500多个,但是我只能想到一个。那么这
\end{quote}

Write a program to find it.
Solution: \url{http://thinkpython2.com/code/cartalk1.py}.

编写一个程序,找到这个单词。答案: http://thinkpython2.com/code/cartalk1.py 。

\end{exercise}


\begin{exercise}
Here's another {\em Car Talk}
Puzzler (\url{http://www.cartalk.com/content/puzzlers}):

下面是另一个来自 *《Car Talk》* 的谜题( http://www.cartalk.com/content/puzzlers ):
\index{Car Talk}  \index{Puzzler}
\index{odometer}  \index{palindrome}

\begin{quote}
``I was driving on the highway the other day and I happened to
notice my odometer. Like most odometers, it shows six digits,
in whole miles only. So, if my car had 300,000
miles, for example, I'd see 3-0-0-0-0-0.

Now, what I saw that day was very interesting. I noticed that the
last 4 digits were palindromic; that is, they read the same forward as
backward. For example, 5-4-4-5 is a palindrome, so my odometer
could have read 3-1-5-4-4-5.

One mile later, the last 5 numbers were palindromic. For example, it
could have read 3-6-5-4-5-6.  One mile after that, the middle 4 out of
6 numbers were palindromic.  And you ready for this? One mile later,
all 6 were palindromic!

The question is, what was on the odometer when I first looked?''
\end{quote}


\begin{quote}
    ``有一天,我正在高速公路上开车,我偶然注意到我的里程表。和大多数里程表一样,它只显示6位数字的整数英里数。
    所以,如果我的车开了300,000英里,我能够看到的数字是:3-0-0-0-0-0。

    我当天看到的里程数非常有意思。我注意到后四位数字是回文数;也就是说,正序读和逆序读是一样的。例如,5-4-4-5就是回文数。
    所以我的里程数可能是3-1-5-4-4-5。

    一英里后,后五位数字变成了回文数。例如,里程数可能变成了是3-6-5-4-5-6。又过了一英里后,6位数字的中间四位变成了回文数。
    你相信吗?一英里后,所有的6位数字都变成了回文数。

    那么问题来了,当我第一次看到里程表时,里程数是多少?''
\end{quote}

Write a Python program that tests all the six-digit numbers and prints
any numbers that satisfy these requirements.
Solution: \url{http://thinkpython2.com/code/cartalk2.py}.

编写写一个程序,测试所有的6位数字,然后输出所有符合要求的结果。答案: http://thinkpython2.com/code/cartalk2.py 。

\end{exercise}


\begin{exercise}
Here's another {\em Car Talk} Puzzler you can solve with a
search (\url{http://www.cartalk.com/content/puzzlers}):

还是 *《Car Talk》* 的谜题( http://www.cartalk.com/content/puzzlers ),你可以通过利用搜索模式解答:
\index{Car Talk}  \index{Puzzler}  \index{palindrome}

\begin{quote}
``Recently I had a visit with my mom and we realized that
the two digits that make up my age when reversed resulted in her
age. For example, if she's 73, I'm 37. We wondered how often this has
happened over the years but we got sidetracked with other topics and
we never came up with an answer.

When I got home I figured out that the digits of our ages have been
reversible six times so far. I also figured out that if we're lucky it
would happen again in a few years, and if we're really lucky it would
happen one more time after that. In other words, it would have
happened 8 times over all. So the question is, how old am I now?''

\end{quote}

\begin{quote}

    ``最近我探望了我的妈妈,我们忽然意识到把我的年纪数字反过来就是她的年龄。比如,如果她73岁,那么我就是37岁。
    我们想知道过去这些年来,发生了多少次这样的巧合,但是我们很快偏离到其他话题上,最后并没有找到答案。

    回到家后,我计算出我的年龄数字有6次反过来就是妈妈的年龄。
    同时,我也发现如果幸运的话,将来几年还可能发生这样的巧合,
    运气再好点的话,之后还会出现一次这样的巧合。
    换句话说,这样的巧合一共会发生8次。那么,问题来了,我现在多大了?''

\end{quote}

Write a Python program that searches for solutions to this Puzzler.
Hint: you might find the string method {\tt zfill} useful.

Solution: \url{http://thinkpython2.com/code/cartalk3.py}.

编写一个查找谜题答案的Python函数。提示:字符串的 ``zfill`` 方法特别有用。
答案:\ http://thinkpython2.com/code/cartalk3.py \ 。


\end{exercise}
